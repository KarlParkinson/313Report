\documentclass[12pt,letterpaper]{article}
\usepackage{setspace}
\usepackage{fancyhdr}

\fancypagestyle{firststyle}
{
  \fancyhf{}
  \fancyhead[L]{Karl Parkinson \\ Jesse Huard \\ CMPUT 313}
  \fancyfoot[C]{\thepage}
}

\thispagestyle{firststyle}

\begin{document}
\onehalfspacing

\paragraph{}
\begin{center}
\Large Insert Title Here
\end{center}

\section{Introduction}

\paragraph{}

A peer-to-peer architecture, or P2P, is the usual alternative to an architecture based on the client-server model.
In the client-server model, one or more central machines provide resources and services to client machines. 
The two different machines, the client and the server, are separate in the sense that the server performs all or most of the computation on behalf of the client. 
In contrast, within a P2P network all the machines both provide and request computational resources and thus operate both as client and server. All the machines in a peer-to-peer network can therefore be considered equals or ``peers.''

\paragraph{}

This paper seeks to summarize the findings of our research into peer-to-peer networks. 
It will do so in three parts. First, the major challenges and concepts of P2P networking will be explored; these include network overlays, routing, and resource discovery. 
Second, some example applications of peer-to-peer computing will be discussed. 
Finally, our concluding section will summarize peer-to-peer networking and why it is important.

\section{Core Concepts}

\subsection{Network Overlays}

\paragraph{}

The basic mechanism underlying peer-to-peer networks is the concept of an overlay network. 
An overlay network is as the name suggests: a network of logical connections between nodes in the peer-to-peer network which live on top of existing TCP/IP networks \cite{overlay}. 
Overlay networks can be split into two main categories which have been called structured overlay networks and unstructured overlay networks. 
In a structured overlay network, nodes exist in a specific topology and content is placed at predetermined parts of the network in order to ensure routing and querying within the network can be done in an efficient manner \cite{overlay}. 
In contrast, unstructured peer-to-peer networks do away with any notions of centrality at all. 
Nodes freely enter and exit the network without needing to know any details about the topology \cite{overlay}.

\paragraph{}

Structured P2P systems generally rely on what is called a Distributed Hash Table, or DHT.

\subsection{Routing}

\subsection{Resource Discovery}

\subsection{Distributed Hash Tables}

\section{Applications}

Freenet\\
BitTorrent\\

\section{Conclusion}

\begin{thebibliography}{1}

  \bibitem{overlay}
    http://ieeexplore.ieee.org/xpl/articleDetails.jsp?arnumber=1610546&newsearch=true&queryText=survey%20comparison%20peer-to-peer%20overlay%20network

\end{thebibliography}

\end{document}
