\documentclass[12pt,letterpaper]{article}
\usepackage{setspace}
\usepackage{fancyhdr}
\usepackage{url}

\fancypagestyle{firststyle}
{
  \fancyhf{}
  \fancyhead[L]{Karl Parkinson \\ Jesse Huard \\ CMPUT 313}
  \fancyfoot[C]{\thepage}
}

\thispagestyle{firststyle}

\begin{document}
\onehalfspacing

\paragraph{}
\begin{center}
\Large Peer to Peer Networking
\end{center}

\section{Introduction}

\paragraph{}
A peer-to-peer architecture, or P2P, is the usual alternative to an architecture based on the client-server model.
In the client-server model, one or more central machines provide resources and services to client machines.
The two different machines, the client and the server, are separate in the sense that the server performs all or most of the computation on behalf of the client.
In contrast, within a P2P network all the machines both provide and request computational resources and thus operate both as client and server.
All the machines in a peer-to-peer network can therefore be considered equals or ``peers.''

\paragraph{}
People are interested in P2P networks for many reasons.
They have very interesting scaling and reliability properties.
The distributed nature of P2P networks allows for the use of computing resources which are spread out over large geographic areas.
This is especially useful for researchers and others who wish to harness large amounts of computational power, but cannot afford to build large datacenters.
Many open source P2P applications have as a guiding principle resisting the centralization that comes with the client-server model, and believe that P2P protocols are more respecting of the rights of users.

\section{Core Concepts}

\subsection{Network Overlays}
\paragraph{}
The basic mechanism underlying peer-to-peer networks is the concept of an overlay network.
An overlay network is as the name suggests: a network of logical connections between nodes in the peer-to-peer network which live on top of existing TCP/IP networks \cite{overlay}.
Overlay networks can be split into two main categories which have been called structured overlay networks and unstructured overlay networks.
In a structured overlay network, nodes exist in a specific topology and content is placed at predetermined parts of the network in order to ensure routing and querying within the network can be done in an efficient manner \cite{overlay}.
This control of resource placement allows even request for rare items to be routed efficiently through the network.
In contrast, unstructured peer-to-peer networks do away with any notions of centrality at all.
Nodes freely enter and exit the network without needing to know any details about the topology \cite{overlay}.
Unstructured P2P networks have poor performance when rare resources are requested, but generally good performance when popular, highly replicated resources are searched for \cite{overlay}.

\subsection{Distributed Hash Tables}
\label{DHT}

\paragraph{}
Structured P2P systems often rely on what is called a Distributed Hash table, or DHT.
Distributed Hash Tables are a type of decentralized distributed system in which data is inserted and accessed through the use of a unique key, similar to a hash table \cite{dht}\cite{wiki-dht}.

\paragraph{}
Normally, the keyspace of a DHT is represented by the set of strings of digits of a given length.
A hash function is then chosen which maps both data and participating nodes to keys.
The hash function used must have some way to define the 'closeness' of two keys \cite{dht}, and is usually chosen such that updates to the DHT only change a localized set of keys \cite{wiki-dht}.
This property is desirable in a DHT, as remapping the data in the DHT to new keys requires moving the data to different nodes over the network, which can be bandwidth-intensive \cite{wiki-dht}.

\paragraph{}
Data storage in a DHT is distributed across the participating nodes.
Given some set of data which hashes to key $k$, the node with the 'closest' key will be responsible for storing the data \cite{dht-ietf}.
When a node joins the DHT and is assigned a key, it will contact 'close' nodes and split the responsibility for storing some of their data with itself \cite{dht-ietf}.
When a node leaves the DHT, its data is replicated to surviving 'close' nodes \cite{dht-ietf}

\paragraph{}
Lookup in a DHT is accomplished using key-based routing.
The number of neighbours a node in the DHT needs to know about is weighed against the longest route length desired in the DHT.
Increasing the degree of each node has the effect of decreasing the route length, and vice-versa \cite{wiki-dht}.
Usually, the degree of nodes in the DHT is chosen to provide $O(\log n)$ lookup, where $n$ is the number of nodes in the DHT.

\paragraph{}
DHTs exhibit very desirable properties when used to construct P2P networks.
They are highly scalable, as load is automatically distributed, they are somewhat robust against node failure, and they are self-organizing \cite{dht-ietf}.
However, DHTs also offer some challenges.
Content in a DHT is looked up by key, which causes complications when keyword or content-based searching is required \cite{dht}\cite{dht-ietf}.
As well, DHTs route based on closest key, when it might also be desirable to route based on the distance of nodes in the underlying network \cite{dht}.
Routing based on keys has the effect of being ill-equipped to handle changing congestion in the network or being able to find the path with the least delay.

\subsection{Routing}

\paragraph{}
The routing scheme used in a P2P network is dependent upon whether the network is structured or unstructured.
Unstructured P2P networks use flooding as the routing mechanism \cite{overlay}, generally with some type of hop-count based TTL in order to avoid queries endlessly routing around the network.
In structured P2P networks, a node keeps a small routing table which has addresses and DHT key values of neighbouring nodes in the overlay network \cite{overlay}.
This enables efficient routing, as when a request comes in to a node for key $k$, the node can determine where to rout the request by using the lookup table.
When the data stored at key $k$ is requested, the node receiving the request will forward it to the neighbour with the closest key \cite{dht}.
If the node with the closest key is the node receiving the request, then it is responsible for either storing the data in the request, or retrieving the data with the received key \cite{dht}.

\paragraph{}
The routing mechanism can have a great affect on the performance of a P2P network.
For instance, in networks designed for streaming media \cite{streaming} the routing mechanism can determine both the quality of streams in the network and what happens when a peer is interrupted in the midst of forwarding a stream onto other peers.


\subsection{Resource Discovery}
\paragraph{}
One of the challenges of P2P networking is finding efficient strategies for resource discovery in the network.
Resource discovery encompasses discovery of both other nodes in the network and what resources those nodes have available, which could be particular data, storage capability, or computing power \cite{wiki-p2p}.

\paragraph{}
To enable resource discovery, nodes joining the network require a mechanism through which to advertise a 'resource specification' to the rest of the network \cite{resource}.
This advertisement is easiest in centralized P2P networks, where nodes joining the network can advertise their IP address and resources to the central server, before being added to the overlay network \cite{resource-mobile}\cite{resource}.
However, while this provides easy discovery, the use of a central server is not as scaleable as decentralized solutions, and can lead to a single point of failure in the network.

\paragraph{}
Decentralized solutions are more robust, but have challenges of their own.
These solutions require some scheme to be in place for nodes to be able to join the network.
One such scheme is the use of 'bootstrap' nodes, which are publicly known, and can therefore provide a point of entry into the network \cite{dht-ietf}.
So long as enough 'bootstrap' nodes are specified, this method is robust to node failure and churn.

\paragraph{}
In unstructured P2P networks, resource discovery usually involves the need to flood the network with the request or perform some sort of random walk through the network \cite{resource}\cite{resource-mobile}.
While flooding works, it is very costly, and therefore does not scale well beyond small networks.
On the other hand, while random walks are much more efficient, they are not guaranteed to find the requested resource, and therefore suffer from quality-of-service issues \cite{resource}.

\paragraph{}
Structured P2P networks such as those based on DHTs provide better resource discovery than unstructured networks, when the resource query is exact.
As discussed in Section \ref{DHT}, key-based routing allows us to quickly and efficiently discover resources in the network, with the caveat that we must know what exactly we are looking for.
The drawbacks of structured solutions are that they do not perform well when given 'fuzzy' queries, and have a higher maintenance overhead as nodes join and leave the network, due to their load-balancing nature \cite{resource}\cite{resource-mobile}.

\subsection{Friend-to-friend Networks}

\paragraph{}
Another of the challenges of P2P networking is providing anonymity to the users of the network.
Among the various solutions to this problem, we will discuss friend-to-friend (F2F) networks here.
F2F networks are a special type of private P2P network in which nodes only directly communicate with known trusted nodes \cite{wiki-f2f}.

%\section{Applications}
%Freenet, BitTorrent

\section{Conclusion}

\paragraph{}
P2P networks relate to what we discussed in class in a number of ways.
First and foremost is the development of efficient routing schemes.
The schemes discussed here were flooding and key-based routing using routing tables.
Routing leads into issues around quality of service.
The different routing schemes and network overlay protocols used affect quality of service concerns such as bandwidth use, how to deal with loss of nodes, and how to deal with congestion on the network.
These are similar to concerns discussed in class around packet loss, and congestion management in modern day TCP/IP networks.

\paragraph{}
This paper has summarized our findings on the research of P2P networks.
Peer-to-peer networks are important as an alternative to the client-server model of computing.
When properly designed, they can be highly scalable, fault-tolerant, and resilient.
The major challenges are routing and resource discovery, and these are addressed through the use of Overlay Networks and Distributed Hash Tables.
Many of the challenges we encountered in class when it comes to building efficient computing networks are encountered in peer-to-peer networks, and P2P offers unique, valuable, and interesting solutions.

\bibliography{report}
\bibliographystyle{plain}

\end{document}
